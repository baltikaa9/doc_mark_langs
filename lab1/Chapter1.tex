\chapter{Теоретические основы моделирования аэродинамических течений}\label{chapter1}

\section{Основные понятия ламинарных и турбулентных течений}
Ламинарное течение является наиболее простым и понятным для человека. Ламинарное течение "--- это течение жидкости или газа, при котором вещество перемещается «слоями», которые не перемешиваются друг с другом и движутся параллельно основному потоку. Данный вид течения наблюдается либо в очень вязких жидкостях, либо в очень медленных течениях, либо в очень узких трубках, таких как, например, капилляры в теле человека~\cite{lam_flow}.

Турбулентное течение является намного более сложным и труднообъяснимым явлением, но оно встречается намного чаще в природе. Движение называют турбулентным, если его характеристики (скорость, давление и т.д.) хаотично изменяются и пульсируют в пространстве и во времени. На рисунке~\ref{fig:lam_vs_turb} схематично изображено отличие ламинарного и турбулентного течения. Явление турбулентности до сих пор полностью не изучено, но известно, что одной из причин появления турбулентности является неустойчивость течения.


\begin{figure}[h!]
\begin{center}
\includegraphics[width=0.4\hsize]{fig/lam_vs_turb.png}\\[2mm]
\caption{Ламинарный (а) и турбулентный (б) потоки. В ламинарном потоке (а) слои жидкости или газа движутся без перемешивания и пульсаций. В турбулентном потоке (б) слои хаотично перемешиваются друг с другом}
\label{fig:lam_vs_turb}
\end{center}
\end{figure}

Часто турбулентность определяют как совокупность разномасштабных вихрей. Максимальный размер вихрей близок к характерному линейному масштабу задачи $L$, например, длине хорды профиля крыла или диаметру трубы. Минимальный размер вихрей характеризуется так называемым колмогоровским масштабом $\displaystyle{\eta_k = \left(\frac{\nu^3}{\varepsilon}\right)^{\frac{1}{4}}}$, где $\varepsilon$ "--- диссипация энергии, а $\nu$ "--- кинематическая вязкость. Масштаб $\eta_k$ характеризует линейные размеры структур, на которые вязкость еще оказывает существенное влияние. Такие вихри рассеиваются в тепло. Основная часть энергии переносится с помощью вихрей среднего размера. 

Турбулентным течениям присущи признаки~\cite{turb_reasons}:
\begin{itemize}
    \item нерегулярность "--- течение нерегулярно, случайно и хаотично;
    \item диффузионность "--- в турбулентном течении диффузия выше чем в ламинарном;
    \item высокое число Рейнольдса;
    \item трехмерность "--- турбулентность всегда трехмерная;
    \item диссипативность "--- энергия наиболее мелких вихрей рассеивается в тепло.
\end{itemize}

\section{Число Рейнольдса как характеристика отношения инерционных и вязких сил}

Число Рейнольдса ($Re$) "--- безразмерная величина, которая является соотношением сил инерции, дестабилизирующих течение, и сил вязкости, стабилизирующих течение. Вычисляется по формуле
\begin{equation}
    \label{eq:re_number}
    Re = \frac{v \, d}{\nu} = \frac{\rho \, v \, d}{\mu},
\end{equation}
где $v$ "--- скорость потока, $d$ "--- характерный размер (например, диаметр трубы или длина хорды крыла), $\mu$ "--- динамическая вязкость среды, $\displaystyle{\nu = \frac{\mu}{\rho}}$ "--- кинематическая вязкость среды.

Число Рейнольдса характеризует режим течения жидкостей и газов. При малых числах Рейнольдса большее влияние на поток оказывают вязкостные силы, а при больших "--- инерционные. Существует, так называемое, критическое значение $Re_{cr}$, при превышении которого ламинарный поток теряет свою устойчивость и становится турбулентным. Критическое число Рейнольдса устанавливается опытным путем для различных течений. Например, при течении в круглых трубах критическое значение $Re_{cr} \approx 2300$, а при обтекании профилей крыльев $Re_{cr} \sim 10^6$~\cite{re_crit}. Если $Re$ меньше значения $Re_{cr}$, то сопротивление крыла велико, а подъемная сила мала; если выше, то сопротивление в несколько раз падает, а подъемная сила в несколько раз возрастает~\cite{re_crit_gadetskiy}.

\section{Основные законы аэродинамики и их применение}
\subsection{Закон сохранения массы "--- уравнение неразрывности}\label{sec_ner}
Уравнение неразрывности является математическим описанием закона сохранения массы для потока идеального сжимаемого газа. Рассмотрим элементарную струйку воздуха. Поскольку поверхность трубки тока непроницаема для частиц воздуха, то при установившемся течении через каждое поперечное сечение элементарной струйки в единицу времени будет протекать одна и та же масса воздуха. За одну секунду через сечение струйки $F_1$ пройдет воздух объемом $m_1 = \rho_1\,v_1\,F_1$, где $\rho$ "--- плотность воздуха, $F_1$ "--- площадь поперечного сечения трубки на входе, $v_1$ "--- скорость потока газа на входе в сечение. Если принять, что трубка тока не имеет разрывов, через которые может поступать или уходить воздух, то через сечение струйки $F_2$ за секунду должна выйти масса воздуха $m_2 = \rho_2\,v_2\,F_2$, равная массе $m_1$, вошедшей в трубку~\cite{uravnenie-nerazryvnosti}. Таким образом, секундный массовый перенос воздуха или другого газа через любое сечение струи есть величина постоянная, рассчитываемая по формуле:\begin{equation}\label{eq:uravnenie_nerazryvnosti} \rho\,v \,F = \mathrm{const}.
\end{equation}

Уравнение~\eqref{eq:uravnenie_nerazryvnosti} называется уравнением неразрывности. При маленьких скоростях ($M < 0.3$) воздух почти не сжимается, поэтому плотность воздуха $\rho$ можно считать постоянной. В этом случае уравнение~\eqref{eq:uravnenie_nerazryvnosti} примет следующий вид:
\begin{equation}
\label{eq:uravnenie_nerazryvnosti_2}
v\,F = \mathrm{const}.
\end{equation}

Из уравнения~\eqref{eq:uravnenie_nerazryvnosti_2} можно сделать вывод, что при уменьшении площади поперечного сечения струйки скорость течения воздуха в ней возрастает, а при увеличении "--- падает. Но это справедливо только для скоростей течения, меньших скорости звука ($M < 1$). При $M > 1$ все происходит наоборот, так как важную роль играет сжимаемость воздуха. При уменьшении площади поперечного сечения плотность воздуха увеличивается настолько, что множитель $\rho\,F$ уравнения~\eqref{eq:uravnenie_nerazryvnosti} тоже увеличивается, что приводит к уменьшению скорости потока $v$. Поэтому для сверхзвуковых потоков для увеличения скорости $v$ необходимо увеличивать площадь сечения $F$.

\subsection{Закон сохранения энергии "--- уравнение Бернулли}\label{sec_bern}
Важное место в аэродинамике занимает закон Бернулли, который связывает давление и скорость воздуха в струе.

Уравнение Бернулли является математическим описанием закона сохранения энергии для струйки идеального несжимаемого газа. Закон сохранения энергии описывается формулой
\begin{equation}\label{eq:eq_bernulli}
    E_1 + E_2 + ... = \mathrm{const}.
\end{equation}

% Уравнение~\eqref{eq:eq_bernulli} означает, что внутри трубки тока, когда нет обмена массой и энергией между струей и окружающей ее средой, сумма всех видов энергии в любом сечении струи постоянна~\cite{uravnenie_bernulli}.

Учитывая только кинетическую энергию и энергию силы давления в струе, уравнение~\eqref{eq:eq_bernulli} можно записать в виде
\begin{equation}\label{eq:eq_bernulli_2}
    E = E_k + E_p,
\end{equation}
где $E$ "--- полная энергия в сечении струи, $E_k$ "--- кинетическая энергия движущегося газа, $E_p$ "--- энергия силы давления газа.

Кинетическая энергия характеризует способность движущегося газа производить работу и рассчитывается по формуле
\begin{equation}\label{eq:E_k}
    E_k = \frac{m\,v^2}{2}.
\end{equation}

% Энергия силы давления характеризует способность газа производить работу силой давления, проталкивающей газ через сечение струи, и рассчитывается по формуле
% \begin{equation}\label{eq:E_p}
%     E_p = p\,v\,S = \frac{\rho\,v\,S\,p}{\rho} = \frac{m\,p}{\rho}.
% \end{equation}

% С учетом~\eqref{eq:E_k} и~\eqref{eq:E_p}, уравнение Бернулли можно записать в виде
% \begin{equation}\label{eq:eq_bernulli_3}
%     \frac{m\,v^2}{2} + \frac{m\,p}{\rho} = \mathrm{const}.
% \end{equation}

% Так, как из уравнения~\eqref{eq:uravnenie_nerazryvnosti} следует, что $m = \mathrm{const}$, и для несжимаемого газа $\rho = \mathrm{const}$, то уравнение~\eqref{eq:eq_bernulli_3} можно переписать в виде
% \begin{equation}\label{eq:eq_bernulli_4}
%     \frac{\rho\,v^2}{2} + p = \mathrm{const}, 
% \end{equation}
% где $\rho\,v^2/2$ "--- динамическое давление, $p$ "--- статическое давление.

% Из уравнения~\eqref{eq:eq_bernulli_4} можно сделать вывод, что при увеличении скорости потока увеличивается динамическое давление, и, так как их сумма должна быть постоянной, статическое давление уменьшается.

% \section{Уравнения Навье–Стокса для описания течений}
% Для моделирования движения жидкостей и газов явлений используется уравнение Навье-Стокса. Уравнение Навье-Стокса "--- это нелинейное дифференциальное уравнение в частных производных, описывающее движение вязкой ньютоновской жидкости или газа.  

% В векторном виде уравнение Навье-Стокса записывается следующим образом
% \begin{equation}\label{eq:nav_st}
%     \frac{\partial \vec{v}}{\partial t} + (\vec{v} \cdot \nabla)\vec{v} = - \frac{1}{\rho} \nabla p + \nu \Delta \vec{v} + \vec{F},
% \end{equation}
% где $\nabla$ "--- оператор набла, $\Delta$ "--- оператор Лапласа, $\vec{v}$ "--- векторное поле скорости, $p$ "--- давление, $t$ "--- время, $\nu$ "--- кинематическая вязкость ($\displaystyle{\nu = \frac{\mu}{\rho}}$), $\vec{F}$ "--- внешние силы.

% Для несжимаемого газа уравнение Навье-Стокса следует дополнить уравнением несжимаемости, указанным в формуле
% \begin{equation}\label{eq:neszhim}
%     \nabla \cdot \vec{v} = 0.
% \end{equation}

% Совместное решение уравнений~\eqref{eq:nav_st},~\eqref{eq:neszhim} и уравнения неразрывности~\eqref{eq:uravnenie_nerazryvnosti_2} позволяет получить поле скоростей и давлений в воздухе~\cite{naviestoks2}. Аналитическое решение этих уравнений в силу их нелинейности удается получить только для небольшого количества простых течений, поэтому для моделирования сложных течений используются численные методы.


% \section{Краткий обзор методов расчета турбулентных течений}

% Несмотря на развитие вычислительной техники и численных алгоритмов для решения задач гидродинамики, численное моделирование турбулентности является одной из самых сложных и актуальных проблем механики жидкостей и газов. В отличие от ламинарных течений, моделирование которых является довольно простой задачей, моделирование турбулентных течений вызывает большие трудности. В настоящее время существует несколько подходов для моделирования турбулентных течений. Все эти подходы основаны на манипуляции над уравнениями Навье-Стокса~\eqref{eq:nav_st}, описывающими движение вязкой ньютоновской жидкости в совокупности с уравнением неразрывности
% \begin{equation}
%     \label{eq:nerazr}
%     \nabla \cdot (\rho\vec{v}) = 0.
% \end{equation}

% \subsection{Метод прямого численного моделирования (DNS)}
% Данный метод заключается в непосредственном решении трехмерных нестационарных уравнений Навье-Стокса. В этом подходе используются пространственные сетки и шаги интегрирования по времени, достаточные для разрешения всех существующих вихрей, даже самых мелких. Шаг сетки в данном случае должен быть порядка колмогоровского масштаба $\eta_k = (\frac{\nu^3}{\varepsilon})^{\frac{1}{4}}$. Прямое численное моделирование позволяет получить невероятную детализацию течения. 

% Метод DNS является очень сложным и ресурсозатратным. Вычислительные затраты пропорциональны количеству узлов в каждом направлении и количеству шагов по времени. Это приводит к тому, что на практике метод используется очень редко и только для расчета простых течений при низких числах Рейнольдса.

% \subsection{Метод моделирования крупных вихрей (LES)}
% Метод моделирования крупных вихрей заключается в фильтрации уравнений Навье-Стокса от коротковолновых турбулентных неоднородностей, то есть происходит разделение вихрей на мелкие (меньше размеров
% используемой расчетной сетки) и крупные. Фильтрация происходит с помощью выражения
% \begin{equation}
%     \label{eq:les_filter}
%     \overline{f}(r, t) = \int\limits_V G(r, r') f(r',t)dr'^3, f = \overline{f} + f',
% \end{equation}
% где $G(r,r')$ "--- функция фильтра, $r$ "--- координата рассматриваемой точки потока, $f$ "--- актуальное значение функции, а $\overline{f}$, $f'$ "--- ее отфильтрованное и пульсационное значения соответственно. 


% После фильтрации строится система осредненных уравнений, в рамках которой крупные вихри будут разрешаться точно, а меньшие вихревые структуры моделироваться. Осредненные уравнения замыкаются при помощи <<подсеточной>> модели турбулентности.

% Преимущество данного подхода заключается в том, что описание турбулентных характеристик при помощи подсеточной модели оказывается гораздо более точным, чем
% моделирование всего спектра турбулентных пульсаций, но данный метод является довольно сильнозатратным. Вычислительные затраты особенно существенны в окрестности стенки.

% \subsection{Осредненные по числу Рейнольдса уравнения Навье-Стокса (RANS)}
% Данный метод заключается в осреднении по времени нестационарных уравнений Навье-Стокса по числу Рейнольдса. При этом подразумевается, что временной интервал, по которому производится осреднение, намного больше характерных временных масштабов турбулентности, с одной стороны, и намного меньше характерного макромасштаба времени рассматриваемого течения, с другой. При осреднении уравнений появляется неизвестный тензор рейнольдсовых напряжений $T_{ij} = -\rho\langle v'_i v'_j \rangle$. Из-за этого невозможно решить уравнение Навье-Стокса совместно с уравнением неразрывности. Замыкание полученных уравнений, то есть определение турбулентных напряжений, производится с помощью полуэмпирических моделей турбулентности, описанных далее. В RANS все турбулентные вихри моделируются, а не рассчитываются точно.


\begin{table}[h!]
\caption{Пример файла с результатами вычислений, полученного из вычислительного модуля (значения выбраны случайным образом)}
\label{tab:csv}
\begin{center}
\begin{tabular}{|c|c|c|c|c|c|}
\hline
 x & y & U (скорость) & u (пр. U на ось х) & v (пр. U на ось у) & p (давление) \\
\hline
 0 & 0 & 50 & 10 & 15 & 15 \\
\hline
 0 & 1 & 49 & 11 & 15 & 14 \\
\hline
 1 & 0 & 51 & 12 & 10 & 13 \\
\hline
 % 0 & 1 & 52 & 13 & 15 & 16 \\
% \hline
\end{tabular}
\end{center}
\end{table}

% \begin{figure}[h!]
%   % \centering
%   \begin{minipage}[b]{0.5\textwidth}
%   \centering \textbf{А} \vspace{0.5 true cm}\\
%     \includegraphics[width=\textwidth]{fig/vel_wing_0.png}
%   \end{minipage}
%   % \hfill
%   \begin{minipage}[b]{0.5\textwidth}
%   \centering \textbf{Б} \vspace{0.5 true cm}\\
%     \includegraphics[width=\textwidth]{fig/vel_wing_18.png}
%   \end{minipage}
%   % \caption{График поля скоростей с выключенными линиями тока. А "--- угол атаки $0^\circ$, Б "--- угол атаки $18^\circ$}\label{fig:vel}
% % \end{figure}

% % \begin{figure}[h!]
%   % \centering
%   \begin{minipage}[b]{0.5\textwidth}
%   \centering \textbf{В} \vspace{0.5 true cm}\\
%     \includegraphics[width=\textwidth]{fig/vel_wing_stream_0.png}
%   \end{minipage}
%   % \hfill
%   \begin{minipage}[b]{0.5\textwidth}
%   \centering \textbf{Г}\vspace{0.5 true cm}\\
%     \includegraphics[width=\textwidth]{fig/vel_wing_stream_2.png}
%   \end{minipage}
%   \caption{График поля скоростей. А "--- угол атаки $0^\circ$ с выключенными линиями тока, Б "--- угол атаки $18^\circ$ с выключенными линиями тока, В "--- угол атаки $0^\circ$ с включенными линиями тока, Г "--- угол атаки $18^\circ$ с включенными линиями тока}\label{fig:vel}
% \end{figure}

Книги: \cite{pres}, \cite{en_book}.

Онлайн-источники: \cite{aero}, \cite{bernuli_and_newton}.

Гиперссылки \footnote{\url{http://airfoiltools.com/airfoil/details?airfoil=n0012-il}}