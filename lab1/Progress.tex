\chapter*{Ход выполнения практики}
\addcontentsline{toc}{chapter}{Ход выполнения практики}

\section*{Задания и отметки о выполнении}

\task{Написать обзор по предметной области исследования,
основываясь на научной, учебной и учебно-методической литературе, как на русском, так и на английском языках. Необходимо использовать современную литературу на английском и русском языках по тематике, поиск которой можно осуществлять по библиографическим базам Scopus, WoS, elibrary, ResearchGate, ADS, ЭБС Лань и др.}

\task{Изучить и описать основные принципы аэродинамики.}

\task{Разработать структуру программного комплекса.}

\task{Определить требования к каждому модулю и к приложению в целом.}

\task{Разработать макеты дизайна и UML-диаграммы, описывающие архитектуру системы.}

\task{Реализовать все модули программного комплекса, используя разработанные диаграммы и макеты.}

\task{Подготовить отчет по практике в соответствии с требованиями  нормоконтроля.}

\task{Создать презентацию в соответствии с данным отчетом и устный доклад для публичной защиты.}

\clearpage
\section*{Отзыв о работе обучающегося}

В период прохождения практики Сергей Александрович Щетина спроектировал информационную модель и разработал программный комплекс для моделирования аэродинамических течений, состоящий из трех модулей.

В работе проведен подробный анализ литературы по теме исследования.
Описаны основы аэродинамики, спроектирована структура программного комплекса, определены требования к приложению, созданы диаграммы, описывающие взаимодействие пользователя с приложением и взаимодействие модулей друг с другом и созданы макеты пользовательского интерфейса программного комплекса. Также описан процесс разработки программного комплекса и сравнение с другим приложением.

С. А. Щетина при выполнении производственной практики, преддипломной практики продемонстрировал необходимые теоретические знания и практические навыки на высоком уровне. Все поставленные перед ним задачи были выполнены. Обучающийся овладел навыками проектирования информационной модели и разработки программного обеспечения. Считаю, что С. А. Щетина заслуживает оценки <<отлично>>.

\progressapprov
