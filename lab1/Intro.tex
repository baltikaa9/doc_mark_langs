\intro
% В данной научно-исследовательской работе рассматривается информационная модель программного комплекса для моделирования аэродинамических течений. Рассматривается структура приложения, функционал, пользовательский интерфейс и способы взаимодействия между модулями программного комплекса.

% В данной работе рассматривается разработка программного комплекса для моделирования аэродинамических течений. Основная цель проекта "--- создание полноценного программного продукта, объединяющего в себе три ключевых компонента: ввод данных, вычислительный модуль и визуализацию результатов. Программный комплекс обеспечивает сквозной цикл от постановки задачи до получения наглядных графиков, что делает его удобным инструментом для инженеров и исследователей, не обладающих глубокими знаниями в области численных методов.

% В современном мире для решения задач авиационной промышленности~\cite{Aerodynamics_of_a_multi_element_airfoil_near_ground}, автомобилестроения~\cite{shushurihin2018},  строительства~\cite{podaeva2017}, медицины~\cite{zhu2015} и других отраслей все чаще и чаще применяется математическое моделирование. Это происходит из-за стремительного развития численных методов и моделей, дающих возможность достаточно точно предсказывать поведение реальных течений. Также математическое моделирование является более дешевым и быстрым инструментом для исследования, чем экспериментальные методы.

% Большинство течений, с которыми приходится иметь дело при решении практических задач, являются турбулентными, поэтому задача о моделировании турбулентных течений является актуальной~\cite{duben}. Даже в настоящее время вычислительная техника недостаточно мощна для точного моделирования сложных турбулентных течений с помощью уравнений Навье-Стокса, поэтому существуют различные методы осреднения уравнений для уменьшения вычислительных затрат, точность которых достаточна для практического применения. 

% Современные универсальные платформы, такие как <<COMSOL Multiphysics>>~\cite{ComsolReview}, <<ANSYS Fluent>>~\cite{ansys}, <<OpenFOAM>>~\cite{openfoam} предлагают богатый функционал, но зачастую требуют значительных вычислительных ресурсов, глубоких теоретических знаний и сложной настройки большого числа параметров. Наш комплекс ориентирован на решение узкоспециализированных задач — моделирование потоков вокруг типовых геометрий (профиль крыла, труба и т. п.) — что позволяет значительно упростить пользовательский интерфейс и снизить порог вхождения для специалистов смежных областей.

В процессе проектирования программного комплекса была сделана ставка на создание инструмента, который отличается простотой использования, узкой специализацией и модульной архитектурой. Существующие программные решения для моделирования аэродинамических течений, такие как <<COMSOL Multiphysics>>, <<ANSYS Fluent>>, <<OpenFOAM>> и другие, представляют собой мощные универсальные платформы, способные решать широкий спектр задач. Однако такие системы зачастую требуют глубоких знаний в области численных методов и программирования, высоких вычислительных мощностей и сложного процесса настройки, включающего десятки параметров, которые необходимо задавать вручную. Проектируемое приложение отличается от существующих решений следующими ключевыми особенностями.
\begin{enumerate}[label=\arabic*.]
    \item Возможность интеграции с разными вычислительными модулями, у которых отсутствует удобный пользовательский интерфейс.
    \item Простота использования. Программный комплекс разрабатывается с учетом пользователей, которые не являются экспертами в программировании или численных методах. Интерфейс интуитивно понятен, а процесс ввода данных сведен к минимально необходимым параметрам.
    \item Узкая специализация. Комплекс ориентирован исключительно на моделирование аэродинамических течений для ограниченного числа типовых задач (например, обтекание профиля крыла или трубы). Это позволяет сосредоточиться на предоставлении качественного функционала именно для данных сценариев, не перегружая систему лишними возможностями.
    \item Модульная архитектура. Программа разделена на три независимых модуля (ввод данных, вычисления и визуализация), что позволяет легко обновлять или изменять отдельные части комплекса. Пользователь может использовать каждый модуль по отдельности или в рамках общего рабочего процесса.
\end{enumerate}

% Объектом исследования является программный комплекс, обеспечивающий полный цикл моделирования аэродинамических течений.

% Целью данной выпускной квалификационной работы (ВКР) является проектирование и реализация пользовательского интерфейса для
% интеграции с вычислительным модулем моделирования аэродинамических
% течений, обеспечивающего создание геометрии, задание условий,
% генерацию сетки и визуализацию результатов.

% проектирование информационной модели программного комплекса, который обеспечит выполнение полного цикла моделирования аэродинамических течений, описание функциональных и нефункциональных требований и реализацию самого приложения, включая реализацию его основных модулей, а также подготовку макетов пользовательского интерфейса и ключевых UML-диаграмм, иллюстрирующих взаимодействие модулей.

% Для достижения этой цели были выполнены следующие задачи.
% \begin{enumerate}[label=\arabic*.]
%     \item Поиск необходимой литературы по предметной области.
%     \item Описание основных принципов аэродинамики.
%     \item Анализ теоретических основ моделирования аэродинамических течений.
%     \item Разработка структуры программного комплекса.
%     % \item Определить требования к каждому модулю и к приложению в целом.
%     \item Проектирование взаимодействия между модулями комплекса.
%     \item Разработка макетов интерфейса и UML-диаграмм, описывающих взаимодействие с системой.
%     \item Реализация модуля ввода данных, позволяющего создавать геометрию, задавать начальные и граничные условия и строить сетку.
%     \item Реализация модуля для визуализации результатов моделирования.
%     \item Моделирование аэродинамики симметричного профиля крыла в турбулентном потоке воздуха при разных углах атаки.
%     \item Подготовка отчета по итогам работы в соответствии с требованиями нормоконтроля и подготовить презентацию для публичной защиты работы.
% \end{enumerate}

% Выпускная квалификационная работа состоит из введения, трех глав, заключения и приложения.
% В первой главе приведены определение и основные понятия аэродинамических течений, даны определения числа Рейнольдса, основные законы аэродинамики и проведен краткий обзор основных методов расчета турбулентных течений.
% Во второй главе описана архитектура программного комплекса, определены функциональные и нефункциональные требования к каждому модулю и ко всему приложению, приведены UML-диаграммы, описывающие работу каждого модуля и взаимодействие между ними, а также макеты пользовательского интерфейса программного комплекса.
% В третьей главе описывается реализация модулей программного комплекса и примеры их использования.
% В заключении проанализированы полученные результаты, подведены итоги проделанной работы, получены соответствующие выводы, а также показаны этапы освоения основных компетенций.

