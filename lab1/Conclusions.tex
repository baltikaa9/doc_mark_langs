\conclusion

% Заключение должно содержать перечисление результатов, полученных при
% выполнении работы, а также те выводы, которые вы сделали при ее
% выполнении. Также заключение может содержать предложения,
% рекомендации и перспективы дальнейшего развития темы. 

В процессе выполнения выпускной квалификационной работы была создана информационная модель программного комплекса, предназначенного для моделирования аэродинамических течений. В работе был проведён анализ предметной области: рассмотрены основные понятия ламинарных и турбулентных течений, число Рейнольдса, законы сохранения массы и энергии, уравнения Навье–Стокса, а также современные подходы к моделированию турбулентности. Спроектирована архитектура программного комплекса, включающая три независимых, но взаимодействующих между собой модуля — ввода данных, вычислений и визуализации, приведены требования к модулям программного комплекса, созданы макеты пользовательского интерфейса, а также описана работа каждого модуля и взаимодействие между ними. Также приведена реализация программного комплекса и пример использования.

Для успешного выполнения всех необходимых работ при выполнении ВКР был составлен подробный календарный план по данному проекту (см. рисунки~\ref{fig:gant1} и~\ref{fig:gant2}). Выполнение каждого этапа работ и строгого следования выдерживанию временных рамок на всех этапах выполнения при создании программного комплекса для моделирования аэродинамических течений позволило выполнить проект в срок.

Реализуемый программный комплекс состоит из трех модулей: модуля ввода данных, вычислительного модуля и модуля визуализации. Такая модульная структура позволяет обеспечить гибкость системы, упрощает ее дальнейшую разработку и позволяет легко адаптировать комплекс под конкретные задачи. Модуль ввода данных реализован с помощью графического фреймворка <<PySide6>> и позволяет интерактивно задавать геометрию, параметры среды, начальные и граничные условия, а также генерировать треугольную конечно-элементную сетку с помощью библиотеки <<Gmsh>>. В качестве вычислительного модуля на данный момент используется <<COMSOL Multiphysics>> куда импортируются данные, полученные из предыдущего модуля. Модуль визуализации использует библиотеку <<Matplotlib>> и позволяет строить графики на основе результатов, полученных из вычислительного модуля. На примере задачи обтекания шара показано, что результаты программного комплекса практически не отличаются от результатов <<COMSOL Multiphysics>>, что подтверждает корректность работы приложения.

Предложенный программный комплекс отличается от существующих решений, таких как <<COMSOL Multiphysics>>, <<ANSYS Fluent>>, <<OpenFOAM>> простотой использования, узкой специализацией и доступностью. Он ориентирован на выполнение типовых задач аэродинамического моделирования и будет полезен для студентов, преподавателей и исследователей, занимающихся данной тематикой.

В процессе выполнения выпускной квалификационной работы мною были освоены следующие компетенции.

% \begin{enumerate}[label=\arabic*)]
    УК-1 Способен осуществлять поиск, критический анализ и синтез информации, применять системный подход для решения поставленных задач. 
    Компетенция освоена при анализе множества различных библиографических источников для максимального ознакомления с данной темой, и в главе~\ref{chapter1} выделены соответствующие теме выдержки с указанием всех использованных источников.
    
    УК-2 Способен определять круг задач в рамках поставленной цели и выбирать оптимальные способы их решения, исходя из действующих правовых норм, имеющихся ресурсов и ограничений.
    Компетенция была освоена в процессе составления списка задач для ВКР, определения временных рамок для их решения и четкого следования намеченному графику решения задач (см. рисунки~\ref{fig:gant1} и~\ref{fig:gant2}).
    % Данная компетенция была освоена при определении требований к программному комплексу, описанных в разделе~\ref{treb} и выборе технологий, соответствующих этим требованиям.
    
    УК-3 Способен осуществлять социальное взаимодействие и реализовывать свою роль в команде.
    Компетенция была освоена в процессе коммуникации с руководителем ВКР в письменном и устном формате по вопросам, связанным с программной реализацией и оформлением работы. 
    
    УК-4 Способен осуществлять деловую коммуникацию в устной и письменной формах на государственном языке Российской Федерации и иностранном(ых) языке(ах). 
    Компетенция была освоена в процессе коммуникации с руководителем ВКР в письменном и устном формате по вопросам, связанным с программной реализацией и оформлением работы.
    % Данная компетенция освоена при консультировании с научным руководителем устно, а также по электронной почте в письмах.
    
    УК-6 Способен управлять своим временем, выстраивать и реализовывать траекторию саморазвития на основе принципов образования в течение всей жизни.
    Компетенция освоена в процессе планирования основных задач проекта, выстраивания временных рамок для решения каждой из этих задач для успешной реализации всего проекта. Все задачи и временные затраты на них отображены на диаграмме Ганта (см. рисунок~\ref{gant}).

    ОПК-1 Способен применять естественнонаучные и общеинженерные знания, методы математического анализа и моделирования, теоретического и экспериментального исследования в профессиональной деятельности.
    Компетенция была освоена в процессе изучения теоретических основ моделирования аэродинамических течений (см. главу~\ref{chapter1}).

    ОПК-2 Способен понимать принципы работы современных информационных технологий и программных средств, в том числе отечественного производства, и использовать их при решении задач профессиональной деятельности.
    Для реализации программного комплекса были использованы библиотеки \texttt{pyside6}, \texttt{matplotlib}, \texttt{gmsh}, а в качестве вычислительного модуля использовалось программное обеспечение <<COMSOL Multiphysics>>.

    ОПК-3 Способен решать стандартные задачи профессиональной деятельности на основе информационной и библиографической культуры с применением информационно-коммуникационных технологий и с учетом основных требований информационной безопасности.
    Компетенция была освоена в процессе написания ВКР с учетом требований ГОСТ, которые закреплены в требованиях к оформлению отчетов по НИР и ВКР на кафедре ИСКМ. В процессе написания литературного обзора во введении были использованы информационно-поисковые системы IEEE Xplore, Google Scholar, Sci-Hub и Scopus для поиска необходимой научно-периодической литературы по тематике исследования.

    ОПК-4 Способен участвовать в разработке стандартов, норм и правил, а также технической документации, связанной с профессиональной деятельностью.
    Компетенция была освоена в процессе составления требований к разрабатываемой программе (см. раздел~\ref{treb}) и разработке архитектуры программного комплекса (см. главу~\ref{chapter2}).
    
    ОПК-5 Способен инсталлировать программное и аппаратное обеспечение для информационных и автоматизированных систем.
    Компетенция была освоена в процессе установки редактора исходного кода PyCharm для написания кода программы и необходимых библиотек. Также был инсталлирован программный продукт <<COMSOL Multiphysics>> для проведения вычислений.

    ОПК-6 Способен разрабатывать алгоритмы и программы, пригодные для практического использования, применять основы информатики и программирования к проектированию, конструированию и тестированию программных продуктов.
    Компетенция была освоена в процессе разработки программного комплекса для моделирования аэродинамических течений (см. главу~\ref{chapter3}).

    ОПК-7 Способен применять в практической деятельности основные концепции, принципы, теории и факты, связанные с информатикой.
    Компетенция была освоена в процессе разработки и реализации программного комплекса (см. главу~\ref{chapter3}).
    
    ОПК-8 Способен осуществлять поиск, хранение, обработку и анализ информации из различных источников и баз данных, представлять ее в требуемом формате с использованием информационных, компьютерных и сетевых технологий.
    Компетенция была освоена в процессе написания литературного обзора во введении и главе~\ref{chapter1} были использованы информационно-поисковые системы IEEE Xplore, Google Scholar, Sci-Hub и Scopus для поиска необходимой научной литературы по тематике исследования.
    
    ПК-1 Способен проводить научно-исследовательские и опытно-конструкторские разработки.
    Компетенция была освоена при исследовании моделей турбулентных течений, используемых в вычислительном модуле, описанных в главе~\ref{chapter1} и разработке архитектуры программного комплекса, описанной в главе~\ref{chapter2}.
    
    ПК-2 Способен проводить интеграцию программных модулей и компонент.
    Компетенция была освоена при разработке программного комплекса, состоящего из трех модулей, которые интегрируются друг с другом (см. главы~\ref{chapter2} и~\ref{chapter3}).
    
    ПК-3 Способен разрабатывать тестовые случаи, проводить тестирование и исследовать результаты.
    Компетенция была освоена при тестировании разработанных модулей и решении задачи аэродинамики (см. разделы~\ref{ex1},~\ref{ex2}).
    
    ПК-4 Способен создавать и анализировать требования на разработку программно-информационных систем и подсистем. 
    Компетенция была освоена при создании функциональных и нефункциональных требований к проектируемому программному комплексу, описанных в разделе~\ref{treb}.
    
    ПК-5 Способен осуществлять концептуальное, функциональное и логическое проектирование программно-информационных систем. 
    Данная компетенция освоена при разработке диаграмм вариантов использования модулей программного комплекса, диаграммы компонентов программного комплекса, диаграммы последовательности, показывающей взаимодействие между всеми модулями программного комплекса и макетов пользовательского интерфейса приложения (см. главу~\ref{chapter2}).
% \end{enumerate}

